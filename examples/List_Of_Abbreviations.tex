%Dokumentklasse
\documentclass[a4paper,12pt,tikz]{scrreprt}
\usepackage[left= 2.5cm,right = 2cm, bottom = 4 cm]{geometry}
%\usepackage[onehalfspacing]{setspace}
% ============= Packages =============

% Dokumentinformationen
\usepackage[
	pdftitle={Masterthesis},
	pdfsubject={},
	pdfauthor={Banias Baabe},
	pdfkeywords={},	
	%Links nicht einrahmen
	hidelinks
]{hyperref}


% Standard Packages
\usepackage[utf8]{inputenc}
\usepackage[ngerman]{babel}
\usepackage[T1]{fontenc}
\usepackage{graphicx, subfig}
\graphicspath{{images/}}
\usepackage{fancyhdr}
\usepackage{lmodern}
\usepackage{color}
\usepackage{setspace}
\usepackage{verbatim}
\usepackage{array}
\usepackage{calc}
\usepackage{mwe}
\usepackage{cite}
\usepackage{float}
\usepackage[printonlyused]{acronym}

% Nur für dummy daten
\usepackage{lipsum}
\usepackage{amsmath} % for aligned
%\usepackage{amssymb} % for \mathbb
\usepackage{tikz}
%\usepackage{etoolbox} % for \ifthen
\usepackage{listofitems} % for \readlist to create arrays
\usetikzlibrary{arrows.meta} % for arrow size
\usepackage{pgfplots}
\usepgfplotslibrary{groupplots}
\usepgfplotslibrary{dateplot}
\usepackage[outline]{contour} % glow around text
\contourlength{1.4pt}
\tikzset{>=latex} % for LaTeX arrow head
\usepackage{xcolor}
\colorlet{myred}{red!80!black}
\colorlet{myblue}{blue!80!black}
\colorlet{mygreen}{green!60!black}
\colorlet{myorange}{orange!70!red!60!black}
\colorlet{mydarkred}{red!30!black}
\colorlet{mydarkblue}{blue!40!black}
\colorlet{mydarkgreen}{green!30!black}
\tikzstyle{node}=[thick,circle,draw=myblue,minimum size=22,inner sep=0.5,outer sep=0.6]
\tikzstyle{node in}=[node,green!20!black,draw=mygreen!30!black,fill=mygreen!25]
\tikzstyle{node hidden}=[node,blue!20!black,draw=myblue!30!black,fill=myblue!20]
\tikzstyle{node convol}=[node,orange!20!black,draw=myorange!30!black,fill=myorange!20]
\tikzstyle{node out}=[node,red!20!black,draw=myred!30!black,fill=myred!20]
\tikzstyle{connect}=[thick,mydarkblue] %,line cap=round
\tikzstyle{connect arrow}=[-{Latex[length=4,width=3.5]},thick,mydarkblue,shorten <=0.5,shorten >=1]
\tikzset{ % node styles, numbered for easy mapping with \nstyle
  node 1/.style={node in},
  node 2/.style={node hidden},
  node 3/.style={node out},
}
\def\nstyle{int(\lay<\Nnodlen?min(2,\lay):3)} % map layer number onto 1, 2, or 3

% Verzeichnis: Anstatt Kapitelnummer wird die Nummer angegeben
\usepackage{chngcntr}
\counterwithout{figure}{chapter}
\counterwithout{table}{chapter}


%Generierte Verzeichnisse werden ins Inhaltsverzeichnis aufgeführt
\usepackage{tocbibind}

% zusätzliche Schriftzeichen der American Mathematical Society
\usepackage{amsfonts}
\usepackage{amsmath}

%nicht einrücken nach Absatz
%\setlength{\parindent}{0pt}


% ============= Kopf- und Fußzeile =============
\pagestyle{fancy}
%
\lhead{\nouppercase{\leftmark}}
\chead{}
\rhead{}
%%
\lfoot{}
\cfoot{\thepage}
\rfoot{}
%%
\renewcommand{\headrulewidth}{0.4pt}
\renewcommand{\footrulewidth}{0pt}

% ============= Package Einstellungen & Sonstiges ============= 
%Besondere Trennungen
\hyphenation{De-zi-mal-tren-nung}

\setlength{\parindent}{0pt}



% ============= Dokumentbeginn =============
\onehalfspacing
\begin{document}
%Seiten ohne Kopf- und Fußzeile sowie Seitenzahl
\pagestyle{empty}


\include{content/Deckblatt}

\include{content/Sperrvermerk}

\makeatletter
\let\partbackup\l@part
\renewcommand*\l@part[2]{\partbackup{#1}{}}

\pagenumbering{Roman}
%\include{02_danksagungen}

\include{content/Zusammenfassung}

% Beendet eine Seite und erzwingt auf den nachfolgenden Seiten die Ausgabe aller Gleitobjekte (z.B. Abbildungen), die bislang definiert, aber noch nicht ausgegeben wurden. Dieser Befehl fügt, falls nötig, eine leere Seite ein, sodaß die nächste Seite nach den Gleitobjekten eine ungerade Seitennummer hat. 

\cleardoubleoddpage

% pagestyle für gesamtes Dokument aktivieren
\pagestyle{fancy}

%Inhaltsverzeichnis
\tableofcontents

%Verzeichnis aller Bilder
\listoffigures

\include{content/Abkuerzungsverzeichnis}
%Abkürzungsverzeichnis fehlt
%Formelverzeichnis fehlt

%Verzeichnis aller Tabellen
\listoftables
\newpage
\pagenumbering{arabic}

\pagestyle{fancy}

% Zeilenabstand soll ab Einleitung beginnen, und nicht ab Inhaltsverzeichnis




\chapter*{List of abbreviations}

\begin{acronym}[cv2]
\acro{mape}[MAPE]{Mean Absolute Percentage Error}
\acro{adi}[ADI]{Average Demand Interval}
\acro{cv2}[CV2]{Coefficient of Variation}
\acro{rnn}[RNN]{Recurrent Neural Networks}
\acro{arima}[ARIMA]{Auto-Regressive Integrated Moving Average}
\acro{sarima}[SARIMA]{Seasonal Auto-Regressive Integrated Moving Average}
\acro{lstm}[LSTM]{Long Short-Term Memory}
\acro{det}[DET]{Deta Network}
\acro{abcdefg}[ABCDEFG]{A Super long abbreviation and it has to be the first one in the list}
\acro{tft}[TF]{Temporal Fusion Transformer}
\acro{cnn}[CNN]{Convolutional Neural Networks}
\acro{dfe}[DFE]{Data Freaking Engineering}
\acro{dae}[DaE]{Data Freaking Engineering}
\end{acronym}

\include{content/Einleitung}


\include{content/Grundlagen}

%\include{03_grundlagen}

%\include{06_standdertechnik}

%\include{07_methoden}

%\include{08_ergebnisse}

%\include{09_diskussion}

%Literaturverzeichnis
\newpage
\bibliographystyle{unsrtdin}
\bibliography{Literatur}

\include{content/Erklaerung}

\end{document}